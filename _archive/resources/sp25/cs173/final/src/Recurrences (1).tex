\documentclass{beamer}
\usepackage{tikz}
\usepackage{pifont}
\usepackage{forest}
\usepackage{pgffor}
% \usepackage{algorithm}
\usetikzlibrary{shapes}
% \usetheme{Madrid}
\usefonttheme{serif}

\let\oldemptyset\emptyset
\let\emptyset\varnothing

\newcommand{\xmark}{\ding{55}}%
\newcommand{\blank}{\underbar{\hphantom{aaaaaaaa}}}

\newcommand{\xbeginlgox}{\begin{minipage}{1in}\begin{tabbing}
  \quad\=\qquad\=\qquad\=\qquad\=\qquad\=\qquad\=\qquad\=\kill}
\newcommand{\xendlgox}{\end{tabbing}\end{minipage}}
\newenvironment{algorithm}{\begin{tabular}{|l|}\hline\xbeginlgox}
  {\xendlgox\\\hline\end{tabular}}
% \usepackage{beamer}
\begin{document}

\begin{frame}
  \frametitle{Recurrences: Review}
  \begin{itemize}[<+->]
    \item We can specify functions with a recursive formula:
    \begin{equation*}
      T(n) = \begin{cases}
        c & n \leq B \\
        \langle \text{Formula with smaller arguments to $T$} \rangle & n > B
      \end{cases}
    \end{equation*}
    \item Some examples:
    
    \begin{tabular}{ c | c }
      Recurrence & Closed Formula \\\hline
      $T(n) = \begin{cases}
        1 & n \leq 1 \\
        2T(n - 1)\hphantom{+ 1} & n > 1
      \end{cases}$ & $T(n) = 2^n$ \\\hline 
      $T(n) = \begin{cases}
        1 & n \leq 1 \\
        2T(n/2) + 1 & n > 1
      \end{cases}$ & $T(n) = 2n - 1$ \\ \hline
      $T(n) = \begin{cases}
        0 & n \leq 1 \\
        2T(n/2) + n & n > 1
      \end{cases}$ & $T(n) = n\log(n)$ \\    
     \end{tabular}
  \end{itemize}
\end{frame}

\begin{frame}[t]
  \frametitle{Solving A Recurrence}
  The general strategy for solving a recurrence is a \textcolor{blue}{recurrence tree}.
  \begin{itemize}[<+->]
    \item Each node has a value equal to the \alert{nonrecursive} work done.
    \item Each node has a child for every recursive call it makes.
  \end{itemize}
  \pause
  Example: $T(n) = \begin{cases}
        0 & n \leq 1 \\
        \underbrace{2T(n/2)}_{\text{recursive}} + \underbrace{n}_{\text{nonrecursive}} & n > 1
      \end{cases}$
  
  % SPEEDING UP COMPILATIONS...
  \begin{forest}
      [$T(n)$
        [$T(n/2)$
          [$T(n/4)$
            [$T(1)$, edge=dotted]
            [\dots, edge=dotted]
          ]
          [$T(n/4)$
            [\dots, edge=dotted]
            [\dots, edge=dotted]
          ]
        ]
        [$T(n/2)$
          [$T(n/4)$
            [\dots, edge=dotted]
            [\dots, edge=dotted]
          ]
          [$T(n/4)$
            [\dots, edge=dotted]
            [$T(1)$, edge=dotted]
          ]
        ]
      ]
  \end{forest}\begin{forest}
    [$n$
      [{$2(n/2) = n$}, no edge
        [{$4(n/4) = n$}, no edge
        [{$n(1) = n$}, no edge]
        ]
      ]
    ]
  \end{forest}
  
\end{frame}

\begin{frame}
  \frametitle{Solving A Recurrence}
  \begin{enumerate}[<+->]
    \item At level $k$...
    \begin{itemize}[<+->]
      \item How many nodes are there?
      \item What is the input? 
      \item How much work is done by each node?
      \item What is the total nonrecursive work? ($W(k)$)
    \end{itemize}
    \item What is the height of the tree (i.e. at what level $h$ do we reach the base case input)?
    \item What is the work done by the leaf nodes? $L = [\text{\# of leaf nodes}] \times [\text{work done by each}]$
    \item The total work is $\underbrace{\sum\limits_{k = 0}^{h - 1} W(k)}_{\text{internal work}} + \underbrace{L}_{\text{leaf work}}$
  \end{enumerate}
\end{frame}

\begin{frame}[t]
  \frametitle{Recurrence Practice I}
  \begin{equation*}
    T(n) = \begin{cases}
    c & n \leq 1 \\
    3T(n/2) + n & n > 1
    \end{cases}
  \end{equation*}
  What is $T(n)$ for $n$ a power of 2?
  \pause
  \begin{enumerate}
    \item At level $k$...
    \begin{itemize}
      \item There are \blank nodes.
      \item The input is \blank.
      \item Each node does \blank work.
      \item The total nonrecursive work $W(k)=$ \blank
    \end{itemize}
    \item $h =$ \blank
    \item What is the work done by the leaf nodes? $L = [\text{\# of leaf nodes}] \times [\text{work done by each}] = $ \blank
    \item $\underbrace{\sum\limits_{k = 0}^{h - 1} W(k)}_{\text{internal work}} + \underbrace{L}_{\text{leaf work}} = $ \blank
  \end{enumerate}

  \pause
  Answer: $T(n) = (c + 2)n^{\log_2 3} - 2n$
\end{frame}

\begin{frame}[t]
  \frametitle{Recurrence Practice II}
  \begin{equation*}
    T(n) = \begin{cases}
    c & n \leq 2 \\
    2T(n - 1) + d & n > 2
    \end{cases}
  \end{equation*}
  What is $T(n)$ for $n \geq 2$?

  \pause 

  \vspace{5cm}

  Answer: $T(n) = (c + d)2^{n - 2} - d$
\end{frame}

\begin{frame}[t]
  \frametitle{Recurrence Practice III}
  \begin{equation*}
    T(n) = \begin{cases}
    c & n \leq 1 \\
    T(n/2) + n^2 & n > 1
    \end{cases}
  \end{equation*}
  What is $T(n)$ for $n$ a power of 2?
  \pause
  \vspace{5cm}


  Answer: $T(n) = \frac43(n^2 - 1) + c$
\end{frame}

\begin{frame}
  \frametitle{What about $\Theta$?}
  \begin{itemize}[<+->]
    \item If all we care about if what $T$ is $\Theta$ of, the specific constants don't matter.
    \item In the previous slide, $T(n) = \frac43(n^2 - 1) + c$ is always $\Theta(n^2)$, no matter what $c$ is.
    \item Similar for $T(n) = (c + d)2^{n - 2} - d$ is $\Theta(2^n)$ and $T(n) = (c + 2)n^{\log_2 3} - 2n$ is $\Theta(n^{\log_2 3})$.
  \end{itemize}
  
\end{frame}

\begin{frame}[t]
  \frametitle{Recurrence Practice V}
  What is the Big-Theta running time of \textsc{Shake}?

  \begin{algorithm}
  \textsc{Shake}($A[1..n]$)\+
  \\ if $n \leq 1$:\+
  \\   return $A$\-
  \\ $m \gets \lfloor n / 2 \rfloor$
  \\ $A_1 \gets \textsc{Shake}(A[1..m])$
  \\ $A_2 \gets \textsc{Shake}(A[m+1..n])$
  \\ $B \gets []$
  \\ for $i$ in $1..m$:\+
  \\   $B$.add($A_1[i]$)
  \\   $B$.add($A_2[i]$)\-
  \\ return $B$
  \end{algorithm}

  \vfill
  \vfill
  \vfill

  \pause

  Answer: $\Theta(n\log n)$
\end{frame}

\begin{frame}[t]
  \frametitle{Recurrence Practice VI: HARD}
  \begin{equation*}
    T(n) = \begin{cases}
    c & n \leq 1 \\
    T(n/2) + T(n/3) + T(n/6) + n & n > 1
    \end{cases}
  \end{equation*}
  What is $T(n)$ Big-Theta of?
  \vfill
  \vfill
  \vfill
  \vfill

  \pause

  Answer: $\Theta(n\log n)$
\end{frame}

\begin{frame}
  \frametitle{Questions/Examples}
  \pause
  \pause
  \pause
\end{frame}

\end{document}