\documentclass{beamer}
\usepackage{tikz}
\usepackage{pifont}
\usepackage{forest}
\usepackage{pgffor}
% \usepackage{algorithm}
\usetikzlibrary{shapes}
% \usetheme{Madrid}
\usefonttheme{serif}

\let\oldemptyset\emptyset
\let\emptyset\varnothing

\newcommand{\xmark}{\ding{55}}%
\newcommand{\blank}{\underbar{\hphantom{aaaaaaaa}}}

\newcommand{\xbeginlgox}{\begin{minipage}{1in}\begin{tabbing}
  \quad\=\qquad\=\qquad\=\qquad\=\qquad\=\qquad\=\qquad\=\kill}
\newcommand{\xendlgox}{\end{tabbing}\end{minipage}}
\newenvironment{algorithm}{\begin{tabular}{|l|}\hline\xbeginlgox}
  {\xendlgox\\\hline\end{tabular}}
% \usepackage{beamer}
\begin{document}

\begin{frame}
  \frametitle{Induction!!!}
  \begin{itemize}[<+->]
    \item Fundamental course concept; closely related to recursive definition in programming languages
    \item \textbf{Goal: }Prove a mathematical statement is true for all natural numbers.
    \item Outline: base case, inductive hypothesis, inductive step
    
  \end{itemize}
\end{frame}

\begin{frame}[t]
  \frametitle{Induction Proofs}
  The general strategy for proving a claim by induction is to 
  \begin{enumerate}[(a)]
      \item define the \textcolor{blue}{\textbf{base case(s)}} and show the claim holds for them
      \item state the \textcolor{blue}{\textbf{inductive hypothesis}} assuming that the claim holds true for all $n < k$ ($n \in \mathbb{N}$)
      \item prove that the claim holds for $n = k$ in the rest of the \textcolor{blue}{\textbf{inductive step}}
  \end{enumerate}
  \begin{itemize}[<+->]
    \pause 
    \item You can also do $n \leq k$ and then $n = k+1$.
    \item \textbf{Always use strong induction!} Your inductive hypothesis must hold for \textit{all} values up to $k$.
  \end{itemize}
  
\end{frame}

\begin{frame}
  \frametitle{How many base cases?}
  Make sure to justify all the base cases that are necessary to establish the claim. 
  \begin{itemize}[<+->]
    \pause
      \item What base cases do you need to prove this claim: $\forall n, f_n < 2^n$, where $f_{n+1} = f_n + f_{n-1}$ ?

    \end{itemize}
    
\end{frame}

\begin{frame}[t]
  \frametitle{Induction Example I}
  Prove that the following holds for all natural numbers $n$. \\
  $$\sum^{n}_{i=0} x^i = \frac{x^{n+1}-1}{x-1}$$
  \pause
  \begin{itemize}
    \item Base case(s): For \blank, we have  \\
    \vspace{0.3cm} 
    \item Inductive hypothesis: Suppose $\Bigg[\hphantom{{\blank\blank}}\Bigg]$
    % {\blank\blank\blank}. 
    \vspace{0.3cm} 
    \item Inductive step: Consider $n=$ \blank. We want to show that $\Bigg[\hphantom{{\blank\blank}}\Bigg]$. \\
    \begin{center}.\end{center}
    \begin{center}.\end{center}
    Therefore $\Bigg[\hphantom{{\blank hihi}}\Bigg]$, which is what we needed to show.
  \end{itemize}
\end{frame}

\begin{frame}[t]
  \frametitle{Induction Example I}
  % Prove that the following holds for all natural numbers $n$. \\

  \begin{itemize}
    \item Inductive step \hphantom{\blank\blank hihihi}$\left(\sum\limits^{n}_{i=0} x^i = \frac{x^{n+1}-1}{x-1}\right)$
  \end{itemize}
\end{frame}

\begin{frame}[t]
  \frametitle{Induction Example I (Solution)}
  Prove that the following holds for all natural numbers $n$. \\
  $$\sum^{n}_{i=0} x^i = \frac{x^{n+1}-1}{x-1}$$
  \begin{itemize}
    \item Base case(s): For $n=0$, we have $\sum_{i=0}^n x^i =x^0=1$ and $\frac{x^{n+1}-1}{x-1} = 
\frac{x^{0+1}-1}{x-1} = \frac{ x-1}{x-1} = 1$. So $\sum_{i=0}^n x^i =\frac{x^{n+1}-1}{x-1}$. \\
    \item Inductive hypothesis: Suppose $\sum^{n}_{i=0} x^i = \frac{x^{n+1}-1}{x-1}$ \ for $n = 0, 1, \ldots, k$. 
    \item Inductive step (next slide)
  \end{itemize}
\end{frame}

\begin{frame}[t]
  \frametitle{Induction Example I (Solution)}
  \begin{itemize}
    \item Inductive step: Consider $n= k+1$. We want to show that $\sum^{n}_{i=0} x^i = \frac{x^{n+1}-1}{x-1}$. \\
    $\sum^{k+1}_{i=0} x^i = \sum_{i=0}^{k+1} x^i = x^{k+1} + \sum_{i=0}^k x^i$. \\
    By the inductive hypothesis, $\sum_{i=0}^k x^i = \frac{x^{k+1}-1}{x-1}$. So:
    \begin{align*}
        \sum_{i=0}^{k+1} x^i &= x^{k+1} + \sum_{i=0}^k x^i = 
x^{k+1} + \frac{x^{k+1}-1}{x-1} \\
&= \frac{(x-1)x^{k+1}}{x-1} + \frac{x^{k+1}-1}{x-1} \\
&= \frac{x^{k+2}-x^{k+1}+ x^{k+1}-1}{x-1} \\
&= \frac{x^{k+2}-1}{x-1}. 
    \end{align*}
    Therefore $\sum^{k+1}_{i=0} x^i = \frac{x^{k+2}-1}{x-1}$, which is what we needed to show.
  \end{itemize}
\end{frame}


\begin{frame}[t]
  \frametitle{Inequality Induction}
  \begin{itemize}[<+->]
      \item Very similar to `equality' induction! Don't overthink it! Just manipulating inequalities instead of equations, still primarily algebra
      \item The only `trick' is that you may have to simplify or change to a term to something even smaller/bigger (resp. the inequality) to make your algebra match your `goal'. (Confused? The next example should illuminate this more clearly.)

    \end{itemize}

\end{frame}

% \begin{frame}[t]
%   \frametitle{Inequality Induction Example I}
%   Let $f_n$ be the $n$th Fibonacci number (i.e. $f_{n+1} = f_n + f_{n-1}$). Prove that $f_n \geq (\frac{3}{2})^{n-2}$.
%   \pause
%   \begin{itemize}
%     \item Base case(s): For \blank, we have \blank \\
%     \item Inductive hypothesis: Suppose that $f_n \geq (\frac{3}{2})^{n-2}$ \ for \blank. 
%     \item Inductive step: Consider $n=$ \blank. We want to show that $f_n \geq \left(\frac{3}{2}\right)^{n-2}$. \\
%     $f_{\rule{0.25cm}{0.15pt}} = \blank$ \\
%     By the inductive hypothesis, \blank \\
%     So \blank \\
%     Therefore $f_n \geq \left(\frac{3}{2}\right)^{n-2}$, which is what we needed to show.
\begin{frame}[t]
  \frametitle{Inequality Induction Example I}
  Let $f_n$ be the $n$th Fibonacci number (i.e. $f_{n+1} = f_n + f_{n-1}$). Prove that $f_n \geq (\frac{3}{2})^{n-2}$.

\end{frame}

\begin{frame}[t]
  \frametitle{Inequality Induction Example I}
  Let $f_n$ be the $n$th Fibonacci number (i.e. $f_{n+1} = f_n + f_{n-1}$ where $f_1 = f_2 = 1$). Prove that $f_n \geq (\frac{3}{2})^{n-2}$.
  \begin{itemize}
    \item Base cases: For $n=1$, we have $f_1 = 1$, $\left(\frac{3}{2}\right)^{1-2} = \frac{2}{3}$. $1 \geq \frac{2}{3}$. \\
    For $n=2$, we have $f_2 = 1$, $\left(\frac{3}{2}\right)^{2-2} = 1$. $1 \geq 1$.
    \\
    \item Inductive hypothesis: Suppose that $f_n \geq \left(\frac{3}{2}\right)^{n-2}$ \ for $n = 1, 2, \ldots, k - 1$. 
    \item Inductive step: Consider $n=k$. We want to show that $f_n \geq (\frac{3}{2})^{n-2}$. \\
    $f_{k} = \blank$ \\
    By the inductive hypothesis, we have $f_{k-1} \geq \left(\frac{3}{2}\right)^{k-3}$ and $f_{k-2} \geq \left(\frac{3}{2}\right)^{k-4}$. \\
    So \blank \\
    Therefore $f_n \geq \left(\frac{3}{2}\right)^{n-2}$, which is what we needed to show.
    
    
  \end{itemize}
\end{frame}

\begin{frame}[t]
  \frametitle{Inequality Induction Example I}
  \begin{itemize}
    \item Inductive step: Consider $n=k$. We want to show that $f_n \geq (\frac{3}{2})^{n-2}$. \\
    $f_{k} = f_{k-1} + f_{k-2}$ \\
    By the inductive hypothesis, we have $f_{k-1} \geq \left(\frac{3}{2}\right)^{k-3}$ and $f_{k-2} \geq \left(\frac{3}{2}\right)^{k-4}$. 
    So \begin{align*}
f_n &\geq \left(\frac{3}{2}\right)^{n-3} + \left(\frac{3}{2}\right)^{n-4} \\
&\geq \left(\frac{3}{2}\right)^{n-4}\left(\frac{3}{2} + 1\right) = \left(\frac{3}{2}\right)^{n-4}\left(\frac{5}{2}\right) \\
&\geq \left(\frac{3}{2}\right)^{n-4}\left(\frac{9}{4}\right) = \left(\frac{3}{2}\right)^{n-4}\left(\frac{3}{2}\right)^2 \\
&\geq \left(\frac{3}{2}\right)^{n-2} 
\end{align*} 
    Therefore $f_n \geq \left(\frac{3}{2}\right)^{n-2}$, which is what we needed to show.
  \end{itemize}
\end{frame}

\begin{frame}[t]
  \frametitle{Tree Induction}
  \begin{itemize}[<+->]
      \item Most important detail to remember: induct on \textbf{height of the tree}, \textit{never} the number of nodes/leaves/etc.
      \item Also critical: build your tree from the `bottom-up' (adding root nodes at the top, never by adding leaves)
      \item May have to do casework, e.g. if node has different values depending on number/properties of children
    \end{itemize}

\end{frame}

\begin{frame}[t]
  \frametitle{Important Tree Terminology}
  \begin{itemize}
      \item \textcolor{blue}{\textbf{binary}}: each node has 0, 1, or 2 children
      \item \textcolor{blue}{\textbf{$n$-ary}}: each node has between $0$ and $n$ children
      \item \textcolor{blue}{\textbf{full}}: each node has strictly either 0 or $n$ children
      \item \textcolor{blue}{\textbf{complete}}: every level, except possibly the last, is completely filled
      \item \textcolor{blue}{\textbf{perfect}}: full and complete; all levels filled, all internal nodes have 2 children, all leaves at same depth
    \end{itemize}

\end{frame}

\begin{frame}[t]
  \frametitle{Tree Induction Example}
  Define a Filbert tree to be a binary tree containing 2D points such that:
  \begin{itemize}
    \item Each leaf node contains (3, 1), (-2, -5), or (2,2).
    \item An internal node with one child labeled $(a, b)$ has label $(a+1, b-1)$.
    \item An internal node with two children labeled $(x, y)$ and $(a, b)$ has label $(x+a, y+b)$.

  \end{itemize}
  Prove that the point in the root node of any Filbert tree is on or below the line $x = y$.
\end{frame}

\begin{frame}[t]
  \frametitle{Tree Induction Example}
  Prove that the point in the root node of any Filbert tree is on or below the line $x = y$.
\end{frame}

% \begin{frame}[t]
%   \frametitle{Tree Induction Example}
%   Prove that the point in the root node of any Filbert tree is on or below the line $x = y$.
  
%   Proof by induction on \blank, where \blank is \blank
%   \begin{itemize}
%     \item Base case(s): For Filbert tree where \blank, we have \blank \\
%     \item Inductive hypothesis: Suppose that the point in the root node of any Filbert tree is on or below the line $x = y$ for trees of \blank. 
%     \item Inductive step: Let $T$ be a Filbert tree of \blank. There are \blank cases: \\
%     \vspace{1.5cm}
%     In all cases the root node contains a point on or below $x = y$, which is what we needed to show.
%   \end{itemize}
% \end{frame}

\begin{frame}[t]
  \frametitle{Tree Induction Example (Solution)}
  Prove that the point in the root node of any Filbert tree is on or below the line $x = y$.
  
  Proof by induction on $h$, where $h$ is the height of the tree.
  \begin{itemize}
    \item Base case(s): For Filbert tree where $h=0$, the root node is a leaf and so contains (3, 1), (-2, -5), or (2,2), all of which are on or below the line $x=y$. \\
    \item Inductive hypothesis: Suppose that the point in the root node of any Filbert tree is on or below the line $x = y$ for trees of height $h = 0, 1, \ldots, k-1$ ($k \geq 1$). 
  \end{itemize}
\end{frame}

% \begin{frame}[t]
%   \frametitle{Tree Induction Example (Solution)}
%   \begin{itemize}
%     \item Inductive step: Let $T$ be a Filbert tree of height $k$. There are 2 cases. \\
%     Case 1: The root of $T$ has one child subtree, whose root contains \blank. The root of $T$ contains \blank. By the inductive hypothesis, \blank.  Since \blank, so this point is on or below $x=y$. \\
%     Case 2: The root of $T$ has two child subtrees, whose roots contain \blank. Then the root of $T$ contains \blank. By the inductive hypothesis, \blank. So \blank, so this point is on or below $x=y$. \\
%     In all cases the root node contains a point on or below $x = y$, which is what we needed to show.
%   \end{itemize}
% \end{frame}

\begin{frame}[t]
  \frametitle{Tree Induction Example (Solution)}
  \begin{itemize}
    \item Inductive step: Let $T$ be a Filbert tree of height $k$. There are 2 cases. \\
    Case 1: The root of $T$ has one child subtree, whose root contains $(a, b)$. The root of $T$ contains $(a+1, b-1)$. By the inductive hypothesis, $(a, b)$ is on or below $x = y$, i.e. $b \leq a$. Since $b \leq a$, $b-1 \leq a+1$, so this point is on or below $x=y$. \\
    Case 2: The root of $T$ has two child subtrees, whose roots contain $(x,y)$ and $(a, b)$. Then the root of $T$ contains $(x+a, y+b)$. By the inductive hypothesis, $y \leq x$ and $b \leq a$. So $y+b \leq x+a \implies (x+a, y+b)$ is on or below $x=y$. \\
    In all cases the root node contains a point on or below $x = y$, which is what we needed to show.
  \end{itemize}
\end{frame}

\begin{frame}[t]
  \frametitle{(Another) Inequality Induction Example}
  Prove by induction that for any two lists of nonnegative numbers $(x_1, \ldots, x_n)$ and $(y_1, \ldots, y_n)$, 
  $$\left(\sum_{i = 1}^n x_i^2\right)\left(\sum_{i = 1}^n y_i^2\right) \geq \left(\sum_{i = 1}^n x_iy_i\right)^2$$
  You may use the AM-GM inequality: For any real numbers $a, b \geq 0, \frac{a + b}{2} \geq \sqrt{ab}$.
\end{frame}

\begin{frame}[t]
  \frametitle{(Another) Inequality Induction Example}
  
  {\tiny \begin{flushright}
    $\left(\sum\limits_{i = 1}^n x_i^2\right)\left(\sum\limits_{i = 1}^n y_i^2\right) \geq \left(\sum\limits_{i = 1}^n x_iy_i\right)^2$, \\
    $\frac{a + b}{2} \geq \sqrt{ab}$
  \end{flushright}}
\end{frame}

\begin{frame}
  \frametitle{Questions/Examples}
  \pause
  \pause
  \pause
\end{frame}

\end{document}