\documentclass{beamer}
\usepackage{tikz}
\usepackage{pifont}
\usetikzlibrary{shapes}
% \usetheme{Madrid}
\usefonttheme{serif}

\let\oldemptyset\emptyset
\let\emptyset\varnothing

\newcommand{\xmark}{\ding{55}}%

% \usepackage{beamer}
\begin{document}
\begin{frame}
  \frametitle{Review: Sets}
    \begin{itemize}
      \item A \textcolor{blue}{set} is a collection of objects without ordering or duplicates. \pause
      \item The {\bf only} thing you can do with a set is see what's in it! \pause
      \begin{itemize}
        \item $A = \{2, 4, 3, 5, 1\}$ \pause
        \item Is 2 in the set $A$? \pause \checkmark \pause
        \item Is 0 in the set $A$? \pause \xmark \pause
        \item What is the first element of $A$? \pause \textcolor{red}{Invalid question!} \pause
      \end{itemize}
      
      \item 3 standard ways to define sets: \pause
      \begin{enumerate}
        \item \textit{Precisely} describe what they contain in English: Let $S$ denote the set of all positive even integers. \pause
        \item List out their elements (with ellipses if necesary and clear): $S = \{2, 4, 6, ...\}$. \pause
        \item Set builder notation: $S = \{2k \mid \underbrace{k \in \mathbb{Z}, k > 0}_{\text{Condition}}\}$ \pause
      \end{enumerate}

      \item Special notation: $\emptyset = \{\}$, the empty set.
    \end{itemize}
\end{frame}

\begin{frame}
  \frametitle{Review: Sets}

  \begin{itemize}[<+->]
    \item ($A \subseteq B$) $A$ is a subset of $B$: $\forall a \in A, a \in B$
    
    $\{1, 2, 4\} \subseteq \{1, 2, 3, 4\}$
    \item Equality: $A = B \Longleftrightarrow A \subseteq B \text{ and } B \subseteq A$

    \item Union: $A \cup B = \{x \mid x \in A \text{ or } x \in B\}$
    
    $\{\text{milk, cheese, eggs}\} \cup \{\text{yogurt, milk, carrots}\} = \{\text{milk, yogurt, cheese, eggs, carrots}\}$

    \item Intersection: $A \cap B = \{x \mid x \in A \text{ and } x \in B\}$
    
    $\{\text{milk, cheese, eggs}\} \cap \{\text{yogurt, milk, carrots}\} = \{\text{milk}\}$

    \begin{itemize}[<+->]
      \item What can you say about $A, A \cup B, \text{ and } A \cap B$?
      \item $A \cap B \subseteq A \subseteq A \cup B$
    \end{itemize}
    \item Cardinality: Given a (finite) set $A$, $|A|$ is the number of elements in the set.
    \item Cartesian Product: $A \times B = \{(a, b) \mid a \in A, b \in B\}$
    \item Powerset (set of all subsets): $\mathbb{P}(A) = \{S \mid S \subseteq A\}$
  \end{itemize}

\end{frame}

\begin{frame}
  \frametitle{Important Points}
  \begin{itemize}[<+->]
    \item Sets can contain sets! The set $A = \{1, 2, 3\}$ is distinct from the set $B = \{A\} = \{\{1, 2, 3\}\}$
    \item Sets need not have the same ``type" of element: $\{1, (3, \text{``no"}), f\}$ is a perfectly valid set.
  \end{itemize}
\end{frame}

\begin{frame}[t]
  \frametitle{Practice Problems I}

  \begin{enumerate}[<+->]
    \item What is $|A|$, where $A = \{1, 2, 4, 1, 5\}$?
    \item What is $\emptyset \times \{1, 2, 3\}$?
    \item What is $\{\emptyset\} \times \{1, 2, 3\}$?
    \item For sets $A$ and $B$, what is $|A \times B|$, in terms of $|A|$ and $|B|$?
    \item What is $|A \cup B|$?
    \item What is $|\mathbb{P}(A)|$?
    \item For any set $A$, what element is \textit{always} in $\mathbb{P}(A)$?
    \item What is $\mathbb{P}(\{1, 2, 3\})$?
  \end{enumerate}
  
\end{frame}

\begin{frame}
  \frametitle{Set Proofs}
    \begin{block}{Proving Inclusion}
      \fbox{\parbox{0.5\textwidth}{
      Let $a \in A$.
      \vspace{0.5cm}

      (logic...)
      \vspace{0.5cm}
  
      Thus, $a \in B$, so $A \subseteq B$.
      }}
        
    \end{block}
    \vspace{2cm}
    \pause
    How would you prove that $A \not \subseteq B$? How about $A \subsetneqq B$?
\end{frame}

\begin{frame}[t]
  \frametitle{Practice Problems II}
  \begin{enumerate}[<+->]
    \item Prove that $\{6x \mid x \in \mathbb{N}\} \subseteq \{2y \mid y \in \mathbb{N}\}$
    \item Prove that $\{p > 2 \mid p \text{ prime}\} \subseteq \{x : 2 \nmid x \in \mathbb{N}\}$
    \item Prove that $\{x^2 \mid x \in \mathbb{R}\} = \{y \geq 0 \mid y \in \mathbb{R}\}$
  \end{enumerate}
\end{frame}

\begin{frame}
  \frametitle{Counting Elements in Sets}
  \begin{itemize}[<+->]
    \item Number of ways to order $n$ objects: $n!$
    \item Number of ways to order $k$ objects from a set of $n$: $\frac{n!}{(n-k)!}$
    \item (!!!) Number of ways to choose $k$ objects from a collection of $n$: ${n \choose k} = \frac{n!}{k!(n - k)!}$
    \item Stars and Bars: How many ways to partition $n$ identical elements into $k$ bins: ${k + n - 1 \choose k - 1}$
  \end{itemize}
\end{frame}

\begin{frame}[t]
  \frametitle{Practice Problems III}
  \begin{enumerate}[<+->]
    \item In a race with $6$ contestants, how many possible top 3 players are there (order matters)?
    \item How many ways are there to divide $2n$ students into two even groups?
    \item How many 10-bit binary strings are there with exactly 3 0 bits?
    \item How many triples of nonnegative numbers $(x, y, z)$ are there such that $x + y + z = 10$?
    \item Same question but for \textit{positive} numbers.
    \item (Summer 2024 Review Question) Let $A = \{(a, b) \in \mathbb{R}^2 \mid a = 3 - b^2\}, B = \{(x, y) \in \mathbb{R}^2 \mid |x| \geq 1 \text{ or } |y| \geq 1\}$. Prove that $A \subseteq B$.
  \end{enumerate}
\end{frame}

\begin{frame}
  \frametitle{Questions/Examples}
\end{frame}

\end{document}